\section{Results}
The range of statistical properties derived from the monthly dataset, which are relevant to the aims of this study, may be seen in Table \ref{table01}. These \emph{a priori} statistical properties of the time series greatly influence the GLS to produce reliable estimates for the known trends that we added to the data, at a wide range of significance levels (\emph{p}-value), but these and other outcomes vary in a systematic manner. Important variables influential in affecting the GLS model include: the i) initial SD of the time series (after detrending but prior to adding artificial slopes); ii) time series length; iii) the magnitude of the added decadal trend; iv) the measurement precision; the amount of missing values in the time series (hereafter referred to as \%\texttt{NA}); and vi) the type of instrument used to collect the data.
\emph{RWS - Need to mention here something about the normal distribution of the data and show "all_plt1.pdf"}
\emph{AJS: Would you please arrange the vars i-v, above, in order a decreasing importance?}

% AJS: Add this to Table 1's caption: "These values were generated only from data at a precision of \SI{0.001}{\degreeCelsius}."

\subsection{Factors affecting the detection of decadal trends by GLS}
As would be expected, the magnitude of the decadal trend (DT) estimated by the GLS increases in direct proportion to the DT which we added \emph{a priori}. What is especially noteworthy in this analysis is that time series of longer duration more often result in trend estimates converging with the actual trend (fig: all_plt1_no_interp_natural.pdf). This effect is most evident from around 30 years. Furthermore, how well the model trend estimate converges with the actual trend is also very visible in the error estimate associated with the estimate of the trend (fig: all_plt2_no_interp_gro.pdf): models fitted to short time series will always have greater SE compared to longer ones. The strength of this correlation  is \emph{r} = 0.56 (\emph{p} <0.001) and it remains virtually unchanged as the added decadal trend is increased. The \emph{p}-value of the fitted models also vary in relation to time series duration and to the steepness of the decadal trend added to the data (fig: all_plt4_no_interp_natural.pdf). It is usually the longer time series equipped with steeper decadal trends that are able to produce model fits with estimated trends that are statistically significant. Note, however, that this \emph{p}-value tests the null hypothesis that the estimated trend is no different from \si{0}{\degreeCelsius}~dec$^{-1}$ at \emph{p} \leq 0.05), and \emph{not} that the slope is not different from the added trend. Taken together, these outcomes show that although our GLS model can very often result in trend estimates that \emph{approach} the true trend, it is seldom that those estimates are statistically significant in the sense that the estimated trends differ statistically from 0.

The variance of the detrended data is another variable that can affect model fitting, but its only systematic influence concerns the error of the trend estimate. Here, it acts in a manner that is entirely consistent across all \emph{a priori} trends (fig: all_plt7_no_interp_grown.pdf). What we see is that as the variance of the data increases (represented here as standard deviation, SD) the error of the slope estimates increases. Moreover, it does so disproportionately more for time series of shorter duration. Again, as we have seen with the magnitude of the estimated trend that converges to the true trend around 30 years, so too does the initial SD of the data cease to be important in time series of around 3 decades in duration.

The number of \%\texttt{NA} permitted in any of our time series was limited to 15\% per time series. Twenty-five of the 84 time series have fewer than 1\%\texttt{NA}. An additional 45 time series have up to 5\%\texttt{NA}, 10 have up to 10\%\texttt{NA} and 4 have up to 15\%\emph{NA}. The mean number of \texttt{NA} for the data is 2.65\%. The relationship between \%\emph{NA} and the \emph{p}-value of the models is shown in Figure X (fig: non_NA_perc.pdf). At 2.5\% or fewer \texttt{NA} their presence does not have any discernible effect on resultant \emph{p}-values. Progressively increasing the number of \texttt{NA}s above 5\%, however, leads to a drastic improvement of models fitted to series with no or gently increasing decadal trends (these generally have very large \emph{p}-values indicative of very poor fits), and a significant deterioration of models fitted to data with steep decadal trends (for these data, the model generally fitted better at low numbers of \%\emph{NA}s, as suggested by the greater number \emph{p}-values that approach 0.05). In other words, the inclusion of missing values results in time series with no added decadal trend to veer away from \si{0}{\degreeCelsius}~dec$^{-1}$ towards a situation where they may erroneously appear to display a trend. On the other hand, time series that do indeed have a decadal trend tend to produce models that prevent these trends from being correctly estimated.

Regarding the effect that the level of precision that the data are measured at has on the outcome of the GLS models, we see in Figure X (fig: correlations_new.pdf) that decreasing the precision from \SI{0.001}{\degreeCelsius} to \SI{0.01}{\degreeCelsius} has an undetectable effect on any differences in the modeled trends that are produced. The Root Mean Square Error (RMSE) between the slopes estimated from \SI{0.001}{\degreeCelsius} and \SI{0.01}{\degreeCelsius} data is 0.001. The correspondence between the slope estimates when data reported at \SI{0.5}{\degreeCelsius} are subjected to the models decreases to a RMSE of 0.03 (relative to at \SI{0.001}{\degreeCelsius}).

% Rob: Can you please explore the real life consequences of these differences without referring to the median and percentile values?

% The effect of decreasing data measurement precision from \SI{0.001}{\degreeCelsius} to \SI{0.5}{\degreeCelsius} has almost no effect on the 5th, 50th (median) and 95th percentiles. There is an increase in the median \emph{R}\textsuperscript{2} value for the new UTR data of 0.01 (from 0.25 to 0.26) when the data are rounded from \SIrange{0.001}{0.5}{\degreeCelsius}. The same rounding has no effect on the range of the 95th percentile but does increase the 5th percentile \emph{R}\textsuperscript{2} value by 0.01 (from 0.09 to 0.10). Rounding the data to \SI{1.0}{\degreeCelsius} has a much larger effect. The median \emph{R}\textsuperscript{2} value for the new UTR data increase by 0.04 (from 0.25 to 0.29), the 95th percentile increase by 0.03 (from 0.70 to 0.73) and the 5th percentile does not change. Whereas decreasing the data precision of the new UTR time series had a measurable effect on the \emph{R}\textsuperscript{2} value of the linear models fitted to them, it had no effect on the \emph{p} value of those linear models. All trends in the new UTR time series remained significant at $\emph{p} \leq 0.001$ regardless of the level of data precision.

An analysis with a large number of variables as shown here is bound to have a medley of complex interactions between the various statistics being measured; however, much of the range seen in the results of the GLS models can be well explained by the influence of one independent variable, or two operating in concert, as we have shown above.

% Rob: the bits below... I'll look at it once I have seen the actual analysis:

%...there is a large difference in mean lengths between the new and old UTR data and the thermometer data. It is therefore important to measure the effect length has on the other results given above so as to not cause undue bias towards the usefulness of the much longer thermometer data.
% To determine the effect length of a time series had on the statistics seen in Table \ref{table01}, the relationship between these values and length of each time series was fitted to a simple linear model. It was found that the positive relationship was significant (\emph{p}\num{< 0.XX}) with an \emph{R}\textsuperscript{2} of 0.XX. \emph{RWS: Repeat for each statistic.} These relationships are shown in Figure \ref{figure02} etc.
% When questioning which types of data produce larger ranges of $\Delta T$ values (with larger rates of linearity) from time series under 10 years, we see that the range of $\Delta T$ (\emph{R}\textsuperscript{2}) values for thermometer data at \SI{-1.9}{\degreeCelsius}~dec$^{-1}$ (0.28) to \SI{3.5}{\degreeCelsius}~dec$^{-1}$ (0.69) is similar to those for the old UTR data with a range from \SI{-5.0}{\degreeCelsius}~dec$^{-1}$ (0.52) to \SI{0.4}{\degreeCelsius}~dec$^{-1}$ (0.7); however, the new UTR data have by far the largest range from \SI{-7.233}{\degreeCelsius}~dec$^{-1}$ (0.81) to \SI{2.576}{\degreeCelsius}~dec$^{-1}$ (0.29). These very large $\Delta T$ and \emph{R}\textsuperscript{2} values serve as a strong example of the necessity for longer time series.
